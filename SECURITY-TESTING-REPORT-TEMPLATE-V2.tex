\documentclass[11pt,a4paper]{article}

% --- 核心宏包加载(按你提供的模版) ---
\usepackage[utf8]{inputenc}
\usepackage[T1]{fontenc}
\usepackage[margin=2cm]{geometry}
\usepackage{titlesec}
\usepackage{xcolor}
\usepackage{booktabs}
\usepackage{tabularx}
\usepackage{enumitem}
\usepackage{hyperref}
\usepackage{amssymb}
\usepackage{tcolorbox}
\usepackage{colortbl}
\usepackage{pgfgantt}
\usepackage{lastpage}
\usepackage{fancyhdr}
\usepackage{graphicx}
\usepackage{tikz}
\usepackage{float}
\usepackage{longtable}
\usepackage{caption}
\usepackage{booktabs,multirow}

\usepackage{newtxtext,newtxmath}

% --- 颜色定义 ---
\definecolor{statusGreen}{HTML}{28A745}
\definecolor{statusYellow}{HTML}{FFC107}
\definecolor{statusRed}{HTML}{DC3545}
\definecolor{headerBlue}{HTML}{003366}

% --- 页眉页脚设置 ---
\pagestyle{fancy}
\renewcommand{\headrulewidth}{0pt}
% logo 文件在仓库中未检索到时不会报错
\fancyhead[L]{\IfFileExists{logo2.png}{\includegraphics[height=0.9cm]{logo2.png}}{}}
\fancyhead[R]{SA61 Team1}
\fancyfoot[C]{Page \thepage\ of \pageref{LastPage}}

% --- 样式设置 ---
\hypersetup{colorlinks=true, linkcolor=headerBlue}
\titleformat{\section}{\large\bfseries\color{headerBlue}}{}{0em}{}[\titlerule]
\titleformat{\subsection}{\normalsize\bfseries}{}{0em}{}
\newlist{todolist}{itemize}{2}
\setlist[todolist]{label=$\square$, leftmargin=1.5em}
\setlist[itemize]{leftmargin=1.2em}
\setlist[enumerate]{leftmargin=1.2em}

\title{EcoGo Security Testing Report}

\begin{document}

% --- Cover page ---
\begin{titlepage}
    \centering
    \IfFileExists{logo (1) (1).png}{\includegraphics[width=0.3\textwidth]{logo (1) (1).png}}{}
    \vspace*{2cm}

    {\Huge\bfseries EcoGo Security Testing\par}
    {\Huge\bfseries Report\par}
    \vspace{0.5cm}
    {\large\bfseries SA4106 AD Project [2520]\par}
    \vspace{2.5cm}
    {\Large National University of Singapore\par}
    \vfill

    {\Large\bfseries Submitted by:\par}
    {\Large Team 1\par}
    \vspace{1cm}

    \vfill
    {\large \today\par}
\end{titlepage}
\newpage

% --- 头部信息 ---
\begin{flushleft}
    {\LARGE \textbf{Security Testing Report}} \\[0.3cm]
    \begin{tabular}{|l|l|}
        \hline
        \textbf{Report Date}     & \today \\ \hline
        \textbf{Project Phase}    & CI/CD Implementation \\ \hline
        \textbf{Branch}           & feature/cicdfeature \\ \hline
        \textbf{Tools Integrated} & 13 \\ \hline
        \textbf{Document Version} & 2.0 (template V2) \\ \hline
    \end{tabular}
\end{flushleft}

\newpage
\tableofcontents
\newpage

% ==================================================
\section{Executive Summary}

\subsection{Scope}
本报告覆盖 EcoGo(Spring Boot + MongoDB)项目在 GitHub Actions CI/CD 流水线中的安全测试与质量控制能力。报告对齐你们当前流水线中实际集成的 \textbf{13 个工具},并基于已下载的产物(JaCoCo、JMeter、ZAP 占位报告)与仓库代码配置给出可追溯的结论。

\subsection{Key Observations (Evidence-based)}
\begin{tcolorbox}[colback=headerBlue!5!white, colframe=headerBlue, title=Key Observations]
\begin{itemize}
  \item \textbf{覆盖范围}: 流水线覆盖 Lint / SAST / SCA / Build / Container Scan / Coverage / Quality / DAST / Monitoring 全链路(见第 \ref{sec:tools} 节)。
  \item \textbf{覆盖率}: JaCoCo(本地产物)显示行覆盖率 1.91\%,分支覆盖率 0.27\%(见第 \ref{sec:coverage} 节)。该数值不足以支撑关键安全回归。
  \item \textbf{性能}: JMeter(本地产物)在 15,000 次请求下 0\% 错误率,平均响应时间 0.88ms(见第 \ref{sec:perf} 节)。
  \item \textbf{DAST 产物缺失}: 本次 ZAP 报告为占位输出,未包含告警明细(见第 \ref{sec:dast} 节)。
  \item \textbf{配置侧风险}: 代码中存在可确认的配置风险(例如 CORS 过于宽松、JWT 密钥管理策略等),已在第 \ref{sec:issues} 节列出。
\end{itemize}
\end{tcolorbox}

% ==================================================
\section{Tools \& Methodology}
\label{sec:tools}

\subsection{Tool Inventory (13) \& Evidence Map}
\begin{table}[H]
\centering
\small
\begin{tabularx}{\textwidth}{c l l X}
\toprule
\textbf{\#} & \textbf{Tool} & \textbf{Category} & \textbf{Config / Output Evidence} \\
\midrule
1  & Checkstyle & Lint / Code Style & \texttt{pom.xml}(插件)+ workflow;本地未提供检查报告产物(No Data)\\
2  & SpotBugs & SAST & \texttt{pom.xml}(插件)+ workflow;本地未提供 \texttt{spotbugsXml.xml}(No Data)\\
3  & OWASP Dependency Check & SCA & \texttt{pom.xml}(插件)+ workflow + \texttt{owasp-suppressions.xml};本地未提供 DC 报告(No Data)\\
4  & Maven & Build/Test & workflow 构建/测试;日志为主要证据(No Artifact)\\
5  & Docker & Container Build & \texttt{Dockerfile}(已存在)+ workflow;镜像扫描结果见 Trivy(需补齐 SARIF)\\
6  & Trivy & Container Scan & workflow 产出 SARIF 并上传 Security Tab;本地未提供 \texttt{trivy-results.sarif}(No Data)\\
7  & JaCoCo & Coverage & 本地已提供:\texttt{新建文件夹/coverage-reports/jacoco.csv}、\texttt{jacoco.xml}(OK)\\
8  & SonarQube & Quality/Security & \texttt{sonar-project.properties} + workflow;本地未提供扫描结果导出(No Data)\\
9  & Integration Tests & Testing & workflow + MongoDB service;本地未提供测试报告(No Data)\\
10 & Smoke Tests & Testing & \texttt{.github/scripts/smoke-tests.sh} + workflow;本地未提供执行输出(No Data)\\
11 & JMeter & Performance & 本地已提供:\texttt{新建文件夹/jmeter-report/statistics.json}(OK)\\
12 & OWASP ZAP & DAST & 本地仅占位:\texttt{新建文件夹/dast-report/zap-report.html}、\texttt{zap-report.md}(Placeholder)\\
13 & Prometheus + Grafana & Monitoring & \texttt{monitoring/*}(配置)+ workflow(部署步骤);运行态指标不在本次产物内\\
\bottomrule
\end{tabularx}
\caption{13 个工具与证据映射(以当前本地可用产物为准)}
\end{table}

\subsection{Pipeline Stages (High-level)}
\begin{tcolorbox}[colback=headerBlue!3!white, colframe=headerBlue, title=Stage Summary]
\small
Lint $\rightarrow$ SAST/SCA $\rightarrow$ Build/Docker $\rightarrow$ Container Scan $\rightarrow$ Coverage $\rightarrow$ SonarQube $\rightarrow$ Deploy $\rightarrow$ Tests $\rightarrow$ Performance $\rightarrow$ DAST $\rightarrow$ Monitoring
\end{tcolorbox}

% ==================================================
\section{Identified Issues}
\label{sec:issues}

\subsection{Issue Summary Table}
\begin{table}[H]
\centering
\small
\begin{tabularx}{\textwidth}{l l l X l}
\toprule
\textbf{ID} & \textbf{Severity} & \textbf{Source} & \textbf{Description} & \textbf{Status} \\
\midrule
ISSUE-001 & High & Code Review & CORS 配置允许通配符 origins/methods/headers 且允许凭证,存在跨域滥用风险。 & Open \\
ISSUE-002 & High & Code Review & JWT 签名密钥在运行时随机生成,非外部化;重启后 token 全失效且不利于密钥轮换与审计。 & Open \\
ISSUE-003 & Medium & Code Review & Docker 容器以 root 身份运行(未设置 \texttt{USER}),容器逃逸/提权面增大。 & Open \\
ISSUE-004 & Medium & Code Review & Spring Security 未显式设置关键安全响应头(CSP/HSTS/XFO 等)。 & Open \\
ISSUE-005 & Medium & Evidence Gap & SAST/SCA/Trivy/Sonar 结果未在本地产物中提供,无法量化漏洞数量与趋势。 & Open \\
ISSUE-006 & Medium & JaCoCo Artifact & 覆盖率偏低(Line 1.91\%,Branch 0.27\%),不利于回归与安全修复验证。 & Open \\
ISSUE-007 & Low & ZAP Artifact & ZAP 报告为占位输出,缺少告警明细,DAST 目前不可审计。 & Open \\
\bottomrule
\end{tabularx}
\caption{Identified Issues(基于代码与本地可用产物)}
\end{table}

\subsection{ISSUE-001: Overly Permissive CORS Configuration}
\textbf{Evidence}: \texttt{src/main/java/com/example/EcoGo/config/SecurityConfig.java}(CORS 配置使用 \texttt{*})。
\begin{tcolorbox}[colback=statusYellow!15!white, colframe=statusYellow, title=Risk]
当前配置允许任意来源发起跨域请求,并允许携带凭证;在浏览器环境下可能导致跨域数据滥用,尤其是当未来引入 Cookie/Session 或允许更多敏感接口跨域时。
\end{tcolorbox}

\subsection{ISSUE-002: JWT Secret Key Management}
\textbf{Evidence}: \texttt{src/main/java/com/example/EcoGo/utils/JwtUtils.java}(\texttt{Keys.secretKeyFor(...)})。
\begin{tcolorbox}[colback=statusYellow!15!white, colframe=statusYellow, title=Risk]
密钥随进程启动随机生成,导致:
\begin{itemize}
  \item 应用重启后所有已发放 token 立即失效(可用性风险)。
  \item 无法通过环境变量/密钥管理系统进行轮换、审计与分环境隔离(安全治理风险)。
\end{itemize}
\end{tcolorbox}

\subsection{ISSUE-003: Container Hardening (Runs as Root)}
\textbf{Evidence}: \texttt{Dockerfile} 未设置 \texttt{USER}。
\begin{tcolorbox}[colback=statusYellow!15!white, colframe=statusYellow, title=Risk]
容器内以 root 身份运行会放大依赖漏洞或 RCE 的影响面。建议创建非 root 用户并以最小权限运行应用。
\end{tcolorbox}

\subsection{ISSUE-004: Missing Security Headers}
\textbf{Evidence}: \texttt{SecurityConfig.java} 未配置 \texttt{http.headers(...)}。
\begin{tcolorbox}[colback=statusYellow!15!white, colframe=statusYellow, title=Risk]
缺少安全响应头可能导致点击劫持、MIME sniffing、弱化浏览器侧防护。建议最少启用 XFO、X-Content-Type-Options、HSTS(生产环境)、合理 CSP。
\end{tcolorbox}

% ==================================================
\section{Fixes Applied}
\label{sec:fixes}

本节描述当前代码库中 \textbf{已经存在} 的安全/质量相关实现(可验证),并对照上一章的 Open issue 给出建议修复方案。

\subsection{Fixes Already Present in Codebase (Verified)}
\begin{table}[H]
\centering
\small
\begin{tabularx}{\textwidth}{l X X}
\toprule
\textbf{Area} & \textbf{What is implemented} & \textbf{Evidence} \\
\midrule
Password hashing & 使用 BCrypt 进行密码加密与校验 & \texttt{src/main/java/.../utils/PasswordUtils.java} \\
Global exception handling & 统一异常返回格式,系统异常返回通用错误码并记录日志 & \texttt{src/main/java/.../exception/GlobalExceptionHandler.java} \\
JWT auth filter chain & 使用 Spring Security + JWT filter 保护接口(认证/鉴权) & \texttt{src/main/java/.../config/SecurityConfig.java} \\
Container health check & Docker 内置健康检查探针(actuator/health) & \texttt{Dockerfile} \\
Multi-stage build & Docker 多阶段构建减少运行镜像内容 & \texttt{Dockerfile} \\
\bottomrule
\end{tabularx}
\caption{Verified fixes/controls already present}
\end{table}

\subsection{Planned Fixes for Identified Issues}
\begin{table}[H]
\centering
\small
\begin{tabularx}{\textwidth}{l X X}
\toprule
\textbf{Issue} & \textbf{Recommended Fix} & \textbf{Re-scan / Verification} \\
\midrule
ISSUE-001 (CORS) & 收敛 origins/methods/headers;生产从环境变量读取允许域名;避免 \texttt{*} + credentials 组合 & 用 ZAP/CORS 预检测试 + 单元/集成测试验证 \\
ISSUE-002 (JWT key) & 将 JWT secret 外部化(ENV/KMS/Secrets),支持轮换;明确 key 长度与算法 & SpotBugs/Sonar 检查硬编码;登录回归测试;重启后 token 行为验证 \\
ISSUE-003 (Docker root) & 创建非 root 用户并设置 \texttt{USER};最小权限运行 & Trivy + 手工验证容器内 UID;镜像基线对比 \\
ISSUE-004 (Headers) & 在 Spring Security 开启安全响应头(XFO、nosniff、HSTS、CSP 等) & ZAP re-scan 期待相关告警清零 \\
ISSUE-005 (Evidence gap) & 将 SpotBugs/DC/Trivy SARIF/Sonar 导出结果作为 artifacts 固化;建立趋势表 & 每次流水线 run 自动更新统计 \\
ISSUE-006 (Coverage) & 增加关键业务路径测试;逐步启用覆盖率门禁 & JaCoCo re-run 目标:Line \(\ge\) 10\%(阶段 1),后续提高 \\
ISSUE-007 (ZAP placeholder) & 调整启动等待/网络/超时;确保输出 JSON/XML 并上传 & ZAP 产物可审计且包含告警列表 \\
\bottomrule
\end{tabularx}
\caption{Fix plan mapping to verification}
\end{table}

% ==================================================
\section{Re-scanning Results}
\label{sec:rescan}

\subsection{Latest Run Status (Based on Local Artifacts)}
\begin{table}[H]
\centering
\small
\begin{tabularx}{\textwidth}{l l X l}
\toprule
\textbf{Tool} & \textbf{Stage} & \textbf{Latest Evidence} & \textbf{Result} \\
\midrule
JaCoCo & Coverage & \texttt{新建文件夹/coverage-reports/jacoco.csv} & Line 1.91\% (Below threshold) \\
JMeter & Performance & \texttt{新建文件夹/jmeter-report/statistics.json} & 15k req, 0\% errors \\
OWASP ZAP & DAST & \texttt{新建文件夹/dast-report/zap-report.html} (placeholder) & No findings exported \\
SpotBugs & SAST & Artifact not provided & No Data \\
OWASP DC & SCA & Artifact not provided & No Data \\
Trivy & Container Scan & SARIF not provided locally & No Data \\
SonarQube & Quality/Security & Export not provided locally & No Data \\
\bottomrule
\end{tabularx}
\caption{Re-scanning status summary}
\end{table}

\subsection{Code Coverage (JaCoCo)}
\label{sec:coverage}
\textbf{Evidence}: \texttt{新建文件夹/coverage-reports/jacoco.csv} 与 \texttt{新建文件夹/coverage-reports/jacoco.xml}.

\begin{table}[H]
\centering
\small
\begin{tabular}{lrrrr}
\toprule
\textbf{Metric} & \textbf{Missed} & \textbf{Covered} & \textbf{Total} & \textbf{Coverage} \\
\midrule
Instructions & 6678 & 432 & 7110 & 6.08\% \\
Branches & 368 & 1 & 369 & 0.27\% \\
Lines & 1589 & 31 & 1620 & 1.91\% \\
Methods & 497 & 50 & 547 & 9.14\% \\
\bottomrule
\end{tabular}
\caption{JaCoCo summary (from downloaded artifact)}
\end{table}

\begin{tcolorbox}[colback=statusYellow!15!white, colframe=statusYellow, title=Coverage Note]
当前覆盖率适合作为 ``pipeline smoke baseline'',但不足以支持关键安全回归。建议将覆盖率门禁从 ``informational'' 逐步提升为 ``enforced''(先对关键模块/关键路径设定阈值)。
\end{tcolorbox}

\subsection{Performance Test (JMeter)}
\label{sec:perf}
\textbf{Evidence}: \texttt{新建文件夹/jmeter-report/statistics.json} 以及 \texttt{新建文件夹/jmeter-report/index.html}.

\begin{table}[H]
\centering
\small
\begin{tabular}{lrrrr}
\toprule
\textbf{Transaction} & \textbf{Samples} & \textbf{Errors} & \textbf{Mean (ms)} & \textbf{Throughput (req/s)} \\
\midrule
Health Check & 5000 & 0 (0\%) & 1.27 & 126.07 \\
Metrics Endpoint & 5000 & 0 (0\%) & 0.72 & 126.20 \\
Info Endpoint & 5000 & 0 (0\%) & 0.67 & 126.22 \\
\midrule
Total & 15000 & 0 (0\%) & 0.88 & 376.31 \\
\bottomrule
\end{tabular}
\caption{JMeter performance summary (from downloaded artifact)}
\end{table}

\subsection{DAST (OWASP ZAP)}
\label{sec:dast}
\textbf{Evidence}: \texttt{新建文件夹/dast-report/zap-report.html} 与 \texttt{zap-report.md}.

\begin{tcolorbox}[colback=statusYellow!15!white, colframe=statusYellow, title=DAST Output Status]
本次下载到本地的 ZAP 报告为占位内容(未生成详细告警列表)。建议优先排查:
\begin{itemize}
  \item 应用在 DAST 步骤前是否已就绪(healthcheck / warm-up)。
  \item ZAP 容器与应用之间的网络连通性(\texttt{--network=host} 在 GitHub Actions runner 的行为)。
  \item 扫描时长与爬虫/主动扫描参数是否需要放宽。
  \item 将 ZAP 的 JSON/XML 报告作为 artifact 固化保存,便于审计与趋势对比。
\end{itemize}
\end{tcolorbox}

% ==================================================
\section{Recommendations}
\subsection{High Priority (Next Sprint)}
\begin{todolist}
  \item 修复 CORS:去通配符并按环境配置允许域名;补充 CORS 预检与回归测试。
  \item JWT 密钥外部化:将 secret 移出代码,接入环境变量/Secrets,并支持轮换策略。
  \item 容器最小权限:Docker 改为非 root 用户运行。
  \item 补齐 SAST/SCA/Trivy/Sonar 产物:将关键报告作为 artifacts 固化,保证可审计与可复盘。
  \item 修复 DAST 产出:确保 ZAP 生成并上传 JSON/XML 报告。
  \item 提升覆盖率:优先补齐 \textbf{service/controller} 的关键业务与鉴权路径测试。
\end{todolist}

\subsection{Mid-term (Next Quarter)}
\begin{todolist}
  \item 引入认证接口限流(防爆破),并将相关指标纳入 Prometheus 告警。
  \item 安全日志(审计)落库与保留策略(最小化敏感信息、可追溯性)。
\end{todolist}

% ==================================================
\section{Appendix: Artifact Map}
\begin{table}[H]
\centering
\small
\begin{tabularx}{\textwidth}{l X}
\toprule
\textbf{Artifact} & \textbf{Local path used in this report} \\
\midrule
JaCoCo coverage & \texttt{新建文件夹/coverage-reports/jacoco.csv}, \texttt{jacoco.xml}, \texttt{index.html} \\
JMeter report & \texttt{新建文件夹/jmeter-report/statistics.json}, \texttt{index.html} \\
ZAP report & \texttt{新建文件夹/dast-report/zap-report.html}, \texttt{zap-report.md} \\
\bottomrule
\end{tabularx}
\caption{Local evidence map}
\end{table}

\end{document}

