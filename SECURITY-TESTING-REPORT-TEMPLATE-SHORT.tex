\documentclass[11pt,a4paper]{article}

% --- 核心宏包加载(按你提供的模版) ---
\usepackage[utf8]{inputenc}
\usepackage[T1]{fontenc}
\usepackage[margin=2cm]{geometry}
\usepackage{titlesec}
\usepackage{xcolor}
\usepackage{booktabs}
\usepackage{tabularx}
\usepackage{enumitem}
\usepackage{hyperref}
\usepackage{amssymb}
\usepackage{tcolorbox}
\usepackage{colortbl}
\usepackage{pgfgantt}
\usepackage{lastpage}
\usepackage{fancyhdr}
\usepackage{graphicx}
\usepackage{tikz}
\usepackage{float}
\usepackage{longtable}
\usepackage{caption}
\usepackage{booktabs,multirow}

\usepackage{newtxtext,newtxmath}

% --- 颜色定义 ---
\definecolor{statusGreen}{HTML}{28A745}
\definecolor{statusYellow}{HTML}{FFC107}
\definecolor{statusRed}{HTML}{DC3545}
\definecolor{headerBlue}{HTML}{003366}

% --- 页眉页脚设置 ---
\pagestyle{fancy}
\renewcommand{\headrulewidth}{0pt}
% logo 文件在仓库中未检索到时不会报错
\fancyhead[L]{\IfFileExists{logo2.png}{\includegraphics[height=0.9cm]{logo2.png}}{}}
\fancyhead[R]{SA61 Team1}
\fancyfoot[C]{Page \thepage\ of \pageref{LastPage}}

% --- 样式设置 ---
\hypersetup{colorlinks=true, linkcolor=headerBlue}
\titleformat{\section}{\large\bfseries\color{headerBlue}}{}{0em}{}[\titlerule]
\titleformat{\subsection}{\normalsize\bfseries}{}{0em}{}
\newlist{todolist}{itemize}{2}
\setlist[todolist]{label=$\square$, leftmargin=1.5em}
\setlist[itemize]{leftmargin=1.2em}
\setlist[enumerate]{leftmargin=1.2em}

\title{EcoGo Security Testing Report}

\begin{document}

% --- Cover page ---
\begin{titlepage}
    \centering
    \IfFileExists{logo (1) (1).png}{\includegraphics[width=0.3\textwidth]{logo (1) (1).png}}{}
    \vspace*{2cm}

    {\Huge\bfseries EcoGo Security Testing\par}
    {\Huge\bfseries Report\par}
    \vspace{0.5cm}
    {\large\bfseries SA4106 AD Project [2520]\par}
    \vspace{2.5cm}
    {\Large National University of Singapore\par}
    \vfill

    {\Large\bfseries Submitted by:\par}
    {\Large Team 1\par}
    \vspace{1cm}

    \vfill
    {\large \today\par}
\end{titlepage}
\newpage

% --- 头部信息 ---
\begin{flushleft}
    {\LARGE \textbf{Security Testing Report}} \\[0.3cm]
    \begin{tabular}{|l|l|}
        \hline
        \textbf{Report Date}     & \today \\ \hline
        \textbf{Project Phase}    & CI/CD Implementation \\ \hline
        \textbf{Branch}           & feature/cicdfeature \\ \hline
        \textbf{Tools Integrated} & 13 \\ \hline
        \textbf{Document Version} & 2.0 (template-short) \\ \hline
    \end{tabular}
\end{flushleft}

\newpage
\tableofcontents
\newpage

% ==================================================
\section{Executive Summary}

\subsection{Scope}
本报告覆盖 EcoGo(Spring Boot + MongoDB)项目在 GitHub Actions CI/CD 流水线中集成的安全与质量控制能力。报告重点对齐你们实际使用的 \textbf{13 个工具},并基于目前已下载的产物(JaCoCo、JMeter、ZAP 占位报告)给出可追溯的指标与改进建议。

\subsection{Key Observations (Evidence-based)}
\begin{tcolorbox}[colback=headerBlue!5!white, colframe=headerBlue, title=Key Observations]
\begin{itemize}
  \item \textbf{Tooling Coverage}: 已覆盖 Lint / SAST / SCA / Container Scan / Coverage / Code Quality / DAST / Monitoring 全链路(见第 \ref{sec:tools} 节)。
  \item \textbf{Test Coverage}: JaCoCo 显示当前覆盖率偏低(见第 \ref{sec:coverage} 节)。
  \item \textbf{Performance}: JMeter 在 15,000 次请求下 0\% 错误率,响应时间处于毫秒级(见第 \ref{sec:perf} 节)。
  \item \textbf{DAST}: 当前下载到本地的 ZAP 报告为占位输出,建议对 DAST 阶段做可用性/连通性校验,确保产出可审计报告(见第 \ref{sec:dast} 节)。
\end{itemize}
\end{tcolorbox}

% ==================================================
\section{Tools \& Pipeline Overview}
\label{sec:tools}

\subsection{Tool Inventory (13)}
\begin{table}[H]
\centering
\small
\begin{tabularx}{\textwidth}{c l l X}
\toprule
\textbf{\#} & \textbf{Tool} & \textbf{Category} & \textbf{Where (repo / pipeline)} \\
\midrule
1  & Checkstyle & Lint / Code Style & \texttt{pom.xml} + \texttt{.github/workflows/cicd-pipeline.yml} \\
2  & SpotBugs & SAST & \texttt{pom.xml} + workflow SAST stage \\
3  & OWASP Dependency Check & SCA & \texttt{pom.xml} + workflow SAST stage \\
4  & Maven & Build/Test & workflow build + test stages \\
5  & Docker & Container Build & \texttt{Dockerfile} + workflow build stage \\
6  & Trivy & Container Scan & workflow container-security stage (SARIF upload) \\
7  & JaCoCo & Coverage & \texttt{pom.xml} + workflow coverage-check stage \\
8  & SonarQube & Quality/Security & workflow sonarqube stage + \texttt{sonar-project.properties} \\
9  & Integration Tests & Testing & workflow integration-tests stages (MongoDB service) \\
10 & Smoke Tests & Testing & workflow smoke-tests stages + \texttt{.github/scripts/smoke-tests.sh} \\
11 & JMeter & Performance & workflow performance-tests stages + \texttt{performance-tests/load-test.jmx} \\
12 & OWASP ZAP & DAST & workflow dast stages + \texttt{.zap/zap-config.yaml} \\
13 & Prometheus + Grafana & Monitoring & workflow monitoring-setup + \texttt{monitoring/*} \\
\bottomrule
\end{tabularx}
\caption{EcoGo CI/CD: tool inventory aligned to workflow configuration}
\end{table}

\subsection{Pipeline Stages (High-level)}
\begin{tcolorbox}[colback=headerBlue!3!white, colframe=headerBlue, title=Stage Summary]
\small
Lint $\rightarrow$ SAST/SCA $\rightarrow$ Build/Docker $\rightarrow$ Container Scan $\rightarrow$ Coverage $\rightarrow$ SonarQube $\rightarrow$ Deploy $\rightarrow$ Tests $\rightarrow$ Performance $\rightarrow$ DAST $\rightarrow$ Monitoring
\end{tcolorbox}

% ==================================================
\section{Results \& Evidence}

\subsection{Code Coverage (JaCoCo)}
\label{sec:coverage}
\textbf{Evidence}: \texttt{新建文件夹/coverage-reports/jacoco.csv} 与 \texttt{新建文件夹/coverage-reports/jacoco.xml}.

\begin{table}[H]
\centering
\small
\begin{tabular}{lrrrr}
\toprule
\textbf{Metric} & \textbf{Missed} & \textbf{Covered} & \textbf{Total} & \textbf{Coverage} \\
\midrule
Instructions & 6678 & 432 & 7110 & 6.08\% \\
Branches & 368 & 1 & 369 & 0.27\% \\
Lines & 1589 & 31 & 1620 & 1.91\% \\
Methods & 497 & 50 & 547 & 9.14\% \\
\bottomrule
\end{tabular}
\caption{JaCoCo summary (from downloaded artifact)}
\end{table}

\begin{tcolorbox}[colback=statusYellow!15!white, colframe=statusYellow, title=Coverage Note]
当前覆盖率适合作为 ``pipeline smoke baseline'',但不足以支持关键安全回归。建议将覆盖率门禁从 ``informational'' 逐步提升为 ``enforced''(先对关键模块/关键路径设定阈值)。
\end{tcolorbox}

\subsection{Performance Test (JMeter)}
\label{sec:perf}
\textbf{Evidence}: \texttt{新建文件夹/jmeter-report/statistics.json} 以及 \texttt{新建文件夹/jmeter-report/index.html}.

\begin{table}[H]
\centering
\small
\begin{tabular}{lrrrr}
\toprule
\textbf{Transaction} & \textbf{Samples} & \textbf{Errors} & \textbf{Mean (ms)} & \textbf{Throughput (req/s)} \\
\midrule
Health Check & 5000 & 0 (0\%) & 1.27 & 126.07 \\
Metrics Endpoint & 5000 & 0 (0\%) & 0.72 & 126.20 \\
Info Endpoint & 5000 & 0 (0\%) & 0.67 & 126.22 \\
\midrule
Total & 15000 & 0 (0\%) & 0.88 & 376.31 \\
\bottomrule
\end{tabular}
\caption{JMeter performance summary (from downloaded artifact)}
\end{table}

\subsection{DAST (OWASP ZAP)}
\label{sec:dast}
\textbf{Evidence}: \texttt{新建文件夹/dast-report/zap-report.html} 与 \texttt{zap-report.md}.

\begin{tcolorbox}[colback=statusYellow!15!white, colframe=statusYellow, title=DAST Output Status]
本次下载到本地的 ZAP 报告为占位内容(未生成详细告警列表)。建议优先排查:
\begin{itemize}
  \item 应用在 DAST 步骤前是否已就绪(healthcheck / warm-up)。
  \item ZAP 容器与应用之间的网络连通性(\texttt{--network=host} 在 GitHub Actions runner 的行为)。
  \item 扫描时长与爬虫/主动扫描参数是否需要放宽。
  \item 将 ZAP 的 JSON/XML 报告作为 artifact 固化保存,便于审计与趋势对比。
\end{itemize}
\end{tcolorbox}

% ==================================================
\section{Recommendations}
\subsection{High Priority (Next Sprint)}
\begin{todolist}
  \item 提升覆盖率:优先补齐 \textbf{service/controller} 的关键业务与鉴权路径测试。
  \item 修复 DAST 产出:确保 ZAP 能生成并上传 JSON/XML 报告,作为安全审计证据。
  \item 将 Trivy SARIF 同步保存为 artifact(目前仅上传 Security Tab,离线复核不便)。
\end{todolist}

\subsection{Mid-term (Next Quarter)}
\begin{todolist}
  \item 引入认证接口限流(防爆破),并将相关指标纳入 Prometheus 告警。
  \item 安全日志(审计)落库与保留策略(最小化敏感信息、可追溯性)。
\end{todolist}

% ==================================================
\section{Appendix: Artifact Map}
\begin{table}[H]
\centering
\small
\begin{tabularx}{\textwidth}{l X}
\toprule
\textbf{Artifact} & \textbf{Local path used in this report} \\
\midrule
JaCoCo coverage & \texttt{新建文件夹/coverage-reports/jacoco.csv}, \texttt{jacoco.xml}, \texttt{index.html} \\
JMeter report & \texttt{新建文件夹/jmeter-report/statistics.json}, \texttt{index.html} \\
ZAP report & \texttt{新建文件夹/dast-report/zap-report.html}, \texttt{zap-report.md} \\
\bottomrule
\end{tabularx}
\caption{Local evidence map}
\end{table}

\end{document}

